\chapter*{Введение}							% Заголовок
\addcontentsline{toc}{chapter}{Введение}	% Добавляем его в оглавление

\newcommand{\actuality}{}
\newcommand{\progress}{}
\newcommand{\aim}{{\textbf\aimTXT}}
\newcommand{\tasks}{\textbf{\tasksTXT}}
\newcommand{\novelty}{\textbf{\noveltyTXT}}
\newcommand{\influence}{\textbf{\influenceTXT}}
\newcommand{\methods}{\textbf{\methodsTXT}}
\newcommand{\defpositions}{\textbf{\defpositionsTXT}}
\newcommand{\reliability}{\textbf{\reliabilityTXT}}
\newcommand{\probation}{\textbf{\probationTXT}}
\newcommand{\contribution}{\textbf{\contributionTXT}}
\newcommand{\publications}{\textbf{\publicationsTXT}}

%\input{common/characteristic} % Характеристика работы по структуре во введении и в автореферате не отличается (ГОСТ Р 7.0.11, пункты 5.3.1 и 9.2.1), потому её загружаем из одного и того же внешнего файла, предварительно задав форму выделения некоторым параметрам

%\textbf{Объем и структура работы.} Диссертация состоит из~введения, трёх глав, заключения и~двух приложений.
%% на случай ошибок оставляю исходный кусок на месте, закомментированным
%Полный объём диссертации составляет  \ref*{TotPages}~страницу с~\totalfigures{}~рисунками и~\totaltables{}~таблицами. Список литературы содержит \total{citenum}~наименований.
%
%Полный объём диссертации составляет
%\formbytotal{TotPages}{страниц}{у}{ы}{}, включая
%\formbytotal{totalcount@figure}{рисун}{ок}{ка}{ков} и
%\formbytotal{totalcount@table}{таблиц}{у}{ы}{}.   Список литературы содержит  
%\formbytotal{citenum}{наименован}{ие}{ия}{ий}.

Объемы данных, собираемых и анализируемых людьми, возрастают.
Кроме того, постоянно повышаются требования к скорости чтения/записи этих данных.
Сами данные, как правило, являются набором характеристик
определенных объектов.
Периодически ставится задача выборки множества объектов,
удовлетворяющих определенным условиям.
К решению этой проблемы можно подходить несколькими способами: во-первых,
масштабировать систему хранения данных --- добавление вычислительных
мощностей должно снижать время обработки и поиска.
Однако такое масштабирование невозможно производить бесконечно как с физической, так и с финансовой точки зрения.
Альтернативный подход --- выбор правильных структур для
хранения данных.
Позволяющих выполнять определенные запросы
максимально быстро и эффективно.

В данной работе рассмотрены возможности применения
структуры данных (индекса) --- кривой Мортона
для решения задачи многокритериального поиска,
проведено сравнение с активно используемыми в настоящее время
B-деревом и R-деревом.
В качестве платформы для реализации была выбрана
нереляционная система управления базами данных Tarantool.
При этом все данные были расположены в оперативной памяти --- in-memory технология.
