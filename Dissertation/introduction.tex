\chapter*{Введение}							% Заголовок
\addcontentsline{toc}{chapter}{Введение}	% Добавляем его в оглавление

\newcommand{\actuality}{}
\newcommand{\progress}{}
\newcommand{\aim}{{\textbf\aimTXT}}
\newcommand{\tasks}{\textbf{\tasksTXT}}
\newcommand{\novelty}{\textbf{\noveltyTXT}}
\newcommand{\influence}{\textbf{\influenceTXT}}
\newcommand{\methods}{\textbf{\methodsTXT}}
\newcommand{\defpositions}{\textbf{\defpositionsTXT}}
\newcommand{\reliability}{\textbf{\reliabilityTXT}}
\newcommand{\probation}{\textbf{\probationTXT}}
\newcommand{\contribution}{\textbf{\contributionTXT}}
\newcommand{\publications}{\textbf{\publicationsTXT}}

%\input{common/characteristic} % Характеристика работы по структуре во введении и в автореферате не отличается (ГОСТ Р 7.0.11, пункты 5.3.1 и 9.2.1), потому её загружаем из одного и того же внешнего файла, предварительно задав форму выделения некоторым параметрам

%\textbf{Объем и структура работы.} Диссертация состоит из~введения, трёх глав, заключения и~двух приложений.
%% на случай ошибок оставляю исходный кусок на месте, закомментированным
%Полный объём диссертации составляет  \ref*{TotPages}~страницу с~\totalfigures{}~рисунками и~\totaltables{}~таблицами. Список литературы содержит \total{citenum}~наименований.
%
%Полный объём диссертации составляет
%\formbytotal{TotPages}{страниц}{у}{ы}{}, включая
%\formbytotal{totalcount@figure}{рисун}{ок}{ка}{ков} и
%\formbytotal{totalcount@table}{таблиц}{у}{ы}{}.   Список литературы содержит  
%\formbytotal{citenum}{наименован}{ие}{ия}{ий}.

Объемы данных, собираемых и анализируемых людьми, постоянно растут. 
Кроме того, постоянно повышаются требования к скорости чтения/записи этих
данных. Сами данные обычно хранятся в структурированном виде, обладая
рядом характеристик, по которым необходимо осуществлять их выбор.
К решению этой проблемы можно подходить несколькими способами: во-первых,
масштабировать систему хранения данных --- добавление вычислительных
мощностей должно снижать время обработки и поиска. Однако, в реальности такое
масштабирование не является выгодным --- оно стоит денег: сначала покупка
оборудования, затем его поддержка.

